\documentclass{beamer}

\mode<presentation> {

\usetheme{Warsaw}

\usecolortheme{orchid}
}

\usepackage{graphicx,tikz,cancel,hyperref}
\usepackage{booktabs,amsmath,array}
\let\olditem\item
\renewcommand{\item}{\setlength{\itemsep}{\fill}\olditem}

\title[SIRV Model]{SIR model with COVID-19 Vaccination}
\author[ B. Martinez, Z. Feng ]{ Bernise Martinez \texorpdfstring{ \\
Faculty Advisor: Dr. Zhaosheng Feng}{}} 

\institute[UTRGV]
{University of Texas Rio Grande Valley \\
\textit{bernise.martinez01@utrgv.edu}}

\date{December 7, 2023}


\begin{document}  % this is where the presentation starts!

\begin{frame}  % Start of a page
\titlepage   
\end{frame}   % End of a page


\begin{frame} 
\frametitle{Introduction}
In 2021, the COVID-19 pandemic continued to shape global health, society, and research efforts. The SIRV (Susceptible-Infectious-Recovered-Vaccinated) model became a crucial tool for understanding and predicting the dynamics of the virus's spread. This year marked a significant phase in the pandemic response, characterized by widespread vaccination campaigns and evolving viral variants.

\end{frame}

\begin{frame} 
\frametitle{Introduction}
Differential equations mathematically express how a function and its derivatives quantitatively relate to each other. They are crucial for modeling and analyzing the dynamic evolution of variables across continuous domains in various scientific disciplines.

\medskip

The Susceptible-Infectious-Recovered-Vaccinated (SIRV) model is a compartmental model widely used in epidemiology to comprehensively capture the dynamics of infectious disease spread within a population.


\end{frame} 


\begin{frame} 
\frametitle{A description of the main result}
The independent variable is time t, measured in days, and we analyze two linked sets of dependent variables as the initial stage in the modeling procedure.
To incorporate vaccinations into the SIR model, 

\begin{itemize}
    \item S = S(t) is the number of susceptible individuals 
    \item I = I(t) is the number of infected individuals
    \item R = R(t) is the number of recovered individuals
    \item V = V(t) is the number of vaccinated individuals
\end{itemize}
\end{frame}   
 

\begin{frame} 
\frametitle{A description of the main result}
If \(N\) is the total population (approximately 7,900,000,000 globally in 2021), we define the following fractions:

    \[ s(t) = \frac{S(t)}{N} \]
    \[ i(t) = \frac{I(t)}{N} \]
    \[ r(t) = \frac{R(t)}{N} \]
    \[ v(t) = \frac{V(t)}{N} \]

\end{frame} 

\begin{frame} 
\frametitle{A description of the main result}
\begin{table}
\begin{center}
\begin{tabular}{|l|l|} \hline 
Parameter/Variable& Description\\ \hline \hline
S& Number of susceptible population\\ \hline 
V&  Number of Vaccinated individuals\\ \hline
 I&Number of Infected population\\\hline
 R&Number of Recovered population\\\hline
 $\beta$&transmission rate\\\hline
 $\gamma$&recovery rate\\\hline
 $\gamma_{\nu}$&vaccination rate \\\hline
 N& total population\\\hline\end{tabular}
\caption{1. State variables and model parameters for the SIRV model. \label{Table}}

\end{center}
\end{table} 
\end{frame} 

\begin{frame} 
\frametitle{Model Equations}
    By dividing each equation by \(N\), we may introduce the fractions \(s(t)\), \(i(t)\), \(r(t)\), and \(v(t)\) and describe the following equations in terms of the percent of the entire population:

\[ \frac{ds}{dt} = -\beta \cdot \frac{SI}{N} - \gamma_{\nu} \cdot S \cdot V\]

\[ \frac{di}{dt} = \beta \cdot \frac{SI}{N}  - \gamma \cdot I \]

\[ \frac{dr}{dt} = \gamma \cdot I \]

\[ \frac{dv}{dt} = \gamma_{\nu} \cdot S \cdot V \]
\end{frame} 

\begin{frame}
\textbf{Parameters:}
\begin{itemize}
    \item $\beta$ = 0.3
    \item $\gamma$ = 0.1
    \item $\gamma_v$ = 0.05
    \item N = 7,900,000,000
\end{itemize}

\medskip

\textbf{Initial Conditions:} (starting with 100 infected individuals)
\begin{itemize}
    \item s = 7,899,999,900
    \item i = 100
    \item r = 0
    \item v = 0
\end{itemize}

\medskip

\textbf{Time Settings:}
T (total time in days) = 365

\smallskip

dt (time step in days) = 1
\end{frame}

\begin{frame}
\frametitle{Differential Equation}
\begin{figure}
\begin{center}
\includegraphics[width=0.95\textwidth]{code.py.pdf}
\end{center}
\end{figure}
\end{frame}

\begin{frame} 
\begin{figure}
\begin{center}
\includegraphics[width=0.95\textwidth]{Figure_3.png}
\end{center}
\end{figure}
\end{frame}


\begin{frame}
\frametitle{Euler's Method}  
Solving the given system of differential equations for the SIRV model involves integrating the equations to obtain the functions \(s(t)\), \(i(t)\), \(r(t)\), and \(v(t)\) over time. 

One commonly used numerical method is Euler's method. Here's a simplified Python code snippet using Euler's method to approximate the solutions:
To create a graph for the given system of differential equations, we need initial conditions and specific parameter values. 
     \[s_{n+1} = s_n + \left(-\beta \cdot \frac{si}{N} - \gamma_{\nu} \cdot s \cdot v\right) \cdot \Delta t\]
     \[i_{n+1} = i_n + \left(\beta \cdot \frac{si}{N} - \gamma \cdot i\right) \cdot \Delta t\]
     \[r_{n+1} = r_n + \left(\gamma \cdot i\right) \cdot \Delta t\]
     \[v_{n+1} = v_n + \left(\gamma_{\nu} \cdot s \cdot v\right) \cdot \Delta t\]

\end{frame}

\begin{frame}
\begin{figure}
\begin{center}
\includegraphics[width=1.0\textwidth]{graph.py.pdf}
\end{center}
\end{figure}
\end{frame}


\begin{frame}  
\begin{figure}
\begin{center}
\includegraphics[width=0.95\textwidth]{Figure_1.png}
\end{center}
\end{figure}
\end{frame}


\begin{frame}
\frametitle{Conclusion and Future Work}
\begin{itemize}
\item  The model allows us to understand how the introduction of a vaccination program affects the dynamics of the disease.
\item The SIRV model helps in assessing the effectiveness of vaccination as a control measure against the spread of COVID-19. 
\item The SIRV model can be employed to project the long-term trends of the epidemic, considering the interplay between natural immunity, waning immunity, and the ongoing vaccination efforts.
\end{itemize}
\end{frame}

\begin{frame}
\frametitle{References}
\footnotesize{
\begin{thebibliography}{99}

\bibitem[Orig]{p1} Turkyilmazoglu, Mustafa (2021)
\newblock  The compartmental SIRVI model
\newblock \emph{An extended epidemic model with vaccination:
Weak-immune SIRVI
}.

\bibitem[Orig]{p1}  Smith, David and Moore, Lang (2004)
\newblock The Differential Equation Model
\newblock \emph{The SIR Model for Spread of Disease}

\bibitem[Orig]{p1}  (2023)
\newblock Population Reference Bureau
\newblock \emph{World Population Data Sheet } 

\end{thebibliography}
}
\end{frame}

\begin{frame}
\Huge{\centerline{The End}}
\end{frame}

\end{document}
